
\chapter{Zakończenie}
Cele opisane we wstępie niniejszej pracy zostały osiągnięte. W ramach projektu została przyswojona podstawowa wiedza z zakresu przetwarzania języka naturalnego oraz budowy i działania modeli uczących, w szczególności sieci neuronowych. Zapoznano się również z narzędziami wykorzystywanymi do realizacji zadań związanych z tematyką pracy. Zdobyta wiedza została wykorzystana w praktyce do przeprowadzenia analizy zbioru danych oraz do stworzenia modeli klasyfikatorów spełniających wymagania czterech podzadań konkursu Sem-Eval. Modele zostały wytrenowane, a następnie przeanalizowane pod kątem skuteczności. Osiągnięto satysfakcjonujące wyniki, modele osiągały skuteczność powyżej 80\% przy zbliżonym rozkładzie ilościowym obu klas.

Podczas realizacji projektu napotkano problemy typowe dla zadań związanych z uczeniem maszynowym. Pierwszy z nich to brak wystarczającej mocy obliczeniowej. Wykorzystywane na początku laptopy i komputery stacjonarne nie posiadały zasobów sprzętowych pozwalających trenować sieci neuronowe w rozsądnym czasie. Konieczne było zwiększenie możliwości sprzętowych przez rozbudowanie komputerów lub skorzystanie z usług platform typu IaaS (Infrastructure as a Service). Drugim problemem była niewystarczająca ilość danych. Problem ten rozwiązano poprzez sztuczne powiększanie zbioru. Lepszym rozwiązaniem byłoby prawdopodobnie zdobycie prawdziwych danych, jednak ze względu na specyfikę projektu nie było to możliwe.

Ponieważ zadania konkursowe są inspirowane rzeczywistymi problemami, stworzone rozwiązania znajdują zastosowania w praktyce. Klasyfikatory porównujące pytania z komentarzami mogłyby posłużyć jako wsparcie dla użytkowników korzystających z forum typu Q\&A. Klasyfikatory badające podobieństwo pytań wsparłyby natomiast administratorów forum. Dostosowanie modeli do działania w ramach rzeczywistego forum otwiera ścieżki dla dalszego rozwoju projektu. Ze względu na ciągły przyrost treści na forum możnaby zaprojektować modele tak, aby było możliwe ich ,,douczanie'' na nowych przykładach (\emph{online learning}). W celu zwiększenia skuteczności możliwe byłoby również stworzenie łańcucha współpracujących klasyfikatorów. Przykładowo, klasyfikator znajdujący podobne pytania mógłby wspomóc działanie klasyfikatora dopasowującego odpowiedzi na nowe pytania.